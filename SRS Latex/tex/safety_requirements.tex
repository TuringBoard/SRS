Include a header paragraph specific to your product here. Safety requirements might address items specific to your product such as: no exposure to toxic chemicals; lack of sharp edges that could harm a user; no breakable glass in the enclosure; no direct eye exposure to infrared/laser beams; packaging/grounding of electrical connections to avoid shock; etc.

\subsection{Laboratory equipment lockout/tagout (LOTO) procedures}
\subsubsection{Description}
Any fabrication equipment provided used in the development of the project shall be used in accordance with OSHA standard LOTO procedures. Locks and tags are installed on all equipment items that present use hazards, and ONLY the course instructor or designated teaching assistants may remove a lock. All locks will be immediately replaced once the equipment is no longer in use.
\subsubsection{Source}
CSE Senior Design laboratory policy
\subsubsection{Constraints}
Equipment usage, due to lock removal policies, will be limited to availability of the course instructor and designed teaching assistants.
\subsubsection{Standards}
Occupational Safety and Health Standards 1910.147 - The control of hazardous energy (lockout/tagout).
\subsubsection{Priority}
Critical

\subsection{National Electric Code (NEC) wiring compliance}
\subsubsection{Description}
Any electrical wiring must be completed in compliance with all requirements specified in the National Electric Code. This includes wire runs, insulation, grounding, enclosures, over-current protection, and all other specifications.
\subsubsection{Source}
CSE Senior Design laboratory policy
\subsubsection{Constraints}
High voltage power sources, as defined in NFPA 70, will be avoided as much as possible in order to minimize potential hazards.
\subsubsection{Standards}
NFPA 70
\subsubsection{Priority}
Critical

\subsection{RIA robotic manipulator safety standards}
\subsubsection{Description}
Robotic manipulators, if used, will either housed in a compliant lockout cell with all required safety interlocks, or certified as a "collaborative" unit from the manufacturer.
\subsubsection{Source}
CSE Senior Design laboratory policy
\subsubsection{Constraints}
Collaborative robotic manipulators will be preferred over non-collaborative units in order to minimize potential hazards. Sourcing and use of any required safety interlock mechanisms will be the responsibility of the engineering team.
\subsubsection{Standards}
ANSI/RIA R15.06-2012 American National Standard for Industrial Robots and Robot Systems, RIA TR15.606-2016 Collaborative Robots
\subsubsection{Priority}
Critical

\subsection{Board response in accidents}
\subsubsection{Description}
The Turing board's response in an accident is necessary. In the regretful case a user should fall off the board, the board must take the appropriate measures to make the user's life easier. There are two main response expected if a user falls off the board. If the user falls of the board and the Turing board is still in the user's vicinity, then the moment the sensor reading changes from "user weight" to "no weight", the Turing board stops. This same mechanism applies in the case the user falls off and lands far away from the user. The board should autonomously find its way back to the user.
\subsubsection{Source}
These responses were specified by the group.
\subsubsection{Constraints}
The board must have a reliable weight sensor that will always return the right weight. A faulty reading could cause an accident itself by suddenly stopping the board.
\subsubsection{Standards}
If the Turing board is within 3 meters of the user after the user falls off, the board will automatically stop.\hfill \break
If the Turing board is outside of a 3 meters parameter of the user after an accident, the board must autonomously roll back to its user. 
\subsubsection{Priority}
Low
\subsection{Kill switch}
\subsubsection{Description}
The role of the Kill switch is to disengage autonomous mode. With the help of the weight sensor, it will be determined if a user, a small load or nothing is on top of the board. When a user is on the Turing board, we must prioritize their safety. The way we will do this is by disengaging autonomous mode and making sure the turning mechanism is locked at 0 degrees. This will make sure the Turing board can be operated as a normal electric long-board and prevent any accidents that could be caused by a misaligned front truck. 
\subsubsection{Source}
The role and mechanics of the kill switch were specified by the team.
\subsubsection{Constraints}
When a set weight is exceeded, meaning a user is on the board, autonomous mode will be disengaged and the turning mechanism will be rotated and locked at 0 degrees for the user's safety.
\subsubsection{Standards}
If a weight greater than 23 kg (50 lbs) is detected, the "kill switch" will be triggered. It will disengage autonomous mode, rotate the trucks back to 0 degrees and the trucks will be locked at that position using solenoids.
\subsubsection{Priority}
Critical