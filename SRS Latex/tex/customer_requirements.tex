Specific requirements for the Turing Board were established with the enjoyment and convenience of the user in mind. In order to have an electronic longboard with autonomous capabilities, the majority of the customer requirements are based on functionality.

\subsection{Forward Propulsion System}
\subsubsection{Description}
Two six-hundred watt brush less DC motor's are=== embedded into the rear truck and will be powered through the on board battery. Utilizing an electronic speed controller, the motors will be driven to a specific RPM that corresponds to the users desired speed.
\subsubsection{Source}
Runtime Terrors
\subsubsection{Constraints}
The forward propulsion system must use a control loop that takes environmental factors into account and adjusts the torque to accommodate any change in variables. Within reason the user should expect to go the same speed regardless of body weight or slight changes to terrain.
\subsubsection{Standards}
Conforms to the Texas state motor-assisted scooter laws.
\subsubsection{Priority}
Critical

\subsection{Turning mechanism}
\subsubsection{Description}
The Turing Board will have a turning mechanism that will be utilized during autonomous mode. Normally, to turn a normal longboard, the user must lean toward the side they want to turn towards. To mitigate and simplify the turning mechanism for the Turing Board during autonomous mode, the board will have a rotating front truck controlled by a stepper motor and a gear box. The turning mechanism will have an inclusive rotation range of -45 to 45 degrees, with 0 degrees being the universal normal front truck position. When the Turing Board is in user mode, the turning mechanism will be rotated back into the normal position and locked at 0 degrees, using solenoids, and the user is in charge of turning the board by leaning toward the desired side, like on normal longboards.
\subsubsection{Source}
Runtime Terrors
\subsubsection{Constraints}
The turning mechanism has a turning range of 90 degrees. With 0 degrees being the origin, the range of this mechanism will be 45 degrees in either directions.
\subsubsection{Standards}
An inclusive turning range of -45 to 45 degrees
\subsubsection{Priority}
Critical

\subsection{Weight sensor}
\subsubsection{Description}
The Weight sensor's main role is to determine if a user, a small load or nothing is on top of the Turing board. With the provided information from the sensor, the appropriate mode will be set. If a user is on the Turing board, then the autonomous mode will be disengaged. This means the turning mechanism will be rotated and locked at 0 degrees. In any other case, a small load or nothing on the Turing board, then the autonomous mode can be engaged if it is needed. 
\subsubsection{Source}
Runtime Terrors
\subsubsection{Constraints}
For small loads, we will have a maximum weight to never exceed. If exceeded, the Turing board will consider it as a person and the autonomous mode will be disengaged. \hfill \break
For user weight sensing, we will set a minimum weight to always exceed in order for the autonomous mode to be disengaged.
\subsubsection{Standards}
0 kg to Turing board's weight - considered "no load" \hfill \break
Turing board's weight to 23 kg (50lbs) - considered "a small load" \hfill \break
23 kg (50lbs) upwards - considered "a person"
\subsubsection{Priority}
Critical

\subsection{GPS}
\subsubsection{Description}
As part of the autonomous navigation system, the GPS will provide coordinate data which will allow the program to know the position of the longboard. This will be used to draw vectors to aid with navigation to a target coordinate. The GPS built into the phone will be used to get the current location of the longboard.
\subsubsection{Source}
Sarker Nadir Afridi Azmi
\subsubsection{Constraints}
The GPS capabilities of the user's phone. The GPS coordinates provided by the phone should be precise without jumping from one value to another too much.
\subsubsection{Standards}
Conforms to the standard Global Positioning System.
\subsubsection{Priority}
High

\subsection{Remote}
\subsubsection{Description}
Instead of having a separate remote, a free-to-download app will be made available which can be used to summon the board, enable the follow-me feature, and control the speed of the wheels when the user is riding it. The app will will be created using React Native to support both Android and iOS. It will use Bluetooth to make the process of data transfer very simple and easy.
\subsubsection{Source}
Runtime Terrors
\subsubsection{Constraints}
Latency of data from remote to board should be minimized. Since the rider of the longboard will be going quite fast, the transmission of data needs to be reduced as much as possible. 
\subsubsection{Standards}
Bluetooth 4.0/5.0 for maximum compatibility.
\subsubsection{Priority}
Critical

\subsection{Computer Vision}
\subsubsection{Description}
The Turing Board utilizes an Intel RealSense Depth Camera D435. It has two that will be used for detecting objects to avoid and also tracking a specific marker in its view to command the board to follow (see "Other" section for more information on this feature). The object detection is handled by Single Shot Detectors (SSDs). An SSD-MobileNet V2 model on a 91 class COCO dataset will be deployed.
\subsubsection{Source}
Sahaj Amatya & Sarker Nadir Afridi Azmi
\subsubsection{Constraints}
The pre-built library for the computer vision only works on x86 architecture. Since our controller, the NVIDIA Jetson TX 2, was built to work with Intel ARM architecture a kernel has been built to allow communication between the two.
Also, due to the nature of what the computer vision of the Turing Board will do, the Jetson TX 2 will need to use it's CUDA cores to process the input appropriately. For this reason, a kernel was also built for OpenCV.
\subsubsection{Standards}
N/A
\subsubsection{Priority}
Critical

\subsection{Path finding}
\subsubsection{Description}
The Turing Board will utilize a Greedy Best First search for its path finding algorithm. The board generates a trapezoidal trajectory map in front of it with progressive width equal to its own width plus some padding width. If an object is detected within the trajectory, the path finding algorithm will seek a path to avoid the object.
\subsubsection{Source}
Sahaj Amatya \& Sarker Nadir Afridi Azmi
\subsubsection{Constraints}
The camera used on the board is able to only see up to 20ft ahead. This will need to be taken into consideration when deciding how to best avoid an object.
\subsubsection{Standards}
N/A
\subsubsection{Priority}
Critical
