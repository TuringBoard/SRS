This section highlights an overview of the Turing board's performance. More precisely, this section will explore task requirements. Further along in this section, each task requirement will be examined in details: its constraints, standards and priority.

The Turing board should be able to follow a target user and it should be able to be summoned from a parking location. The board should be able to use computer vision, the turning mechanism and the propulsion mechanism to navigate. The board should also be able to function as a normal electric long-board when a user is on it.

\subsection{Battery}
\subsubsection{Description}
A 2500 KW hour, 24 V battery will provide power to the entire system. For the preliminary design, the voltage will be stepped down using a Buck Converter providing ~19V to the Jetson TX2. There will be an open 5V and 3.3V terminal if there is a need to connect external power to sensors and microcontrollers.
\subsubsection{Source}
Specified by the team member (Sarker Nadir Afridi Azmi)
\subsubsection{Constraints}
The power used by the entire system should be minimized to ensure the the longboard can operate for at least an hour with a rider before it needs to be charged.
\subsubsection{Standards}
2500 KWh
\subsubsection{Priority}
High